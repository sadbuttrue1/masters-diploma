\begin{titlepage}

\begin{center}
\vspace{1.5em}
\small{МИНИСТЕРСТВО ОБРАЗОВАНИЯ И НАУКИ РОССИЙСКОЙ ФЕДЕРАЦИИ\\
\vspace{\baselineskip}
ФЕДЕРАЛЬНОЕ ГОСУДАРСТВЕННОЕ АВТОНОМНОЕ\\
ОБРАЗОВАТЕЛЬНОЕ УЧРЕЖДЕНИЕ ВЫСШЕГО ОБРАЗОВАНИЯ\\
<<CАМАРСКИЙ НАЦИОНАЛЬНЫЙ ИССЛЕДОВАТЕЛЬСКИЙ УНИВЕРСИТЕТ\\
ИМЕНИ АКАДЕМИКА С. П. КОРОЛЕВА>>\\}
\vspace{\baselineskip}
\end{center}

\begin{minipage}{.45\linewidth}
	\begin{flushleft}                           
	\small{УДК 629.78}
	\end{flushleft} 
\end{minipage}
\hfill
\begin{minipage}{.45\linewidth}
	\begin{flushright}                
	\small{На правах рукописи}
	\end{flushright} 
\end{minipage}

\vspace{3em}

\begin{center}
	Асланов Евгений Владимирович
\end{center}

\begin{center}
Удаление космического мусора путем электростатического взаимодействия с активным космическим аппаратом
\end{center}

\vspace{3em}

\begin{center}
\small{Автореферат\\
выпускной квалификационной работы магистра\\
по направлению подготовки 01.04.03 <<Механика и математическое моделирование>>\\
магистерская программа <<Математическое и компьютерное проектирование механики космических систем>>}
\end{center}

\vspace{\fill}

\begin{center}
Самара 2017
\end{center}
\end{titlepage}
\newpage

\thispagestyle{empty}
Работа выполнена в Самарском национальном исследовательском университете имени академика С. П. Королева.

Научный руководитель: доктор технических наук, профессор Асланов В. С.

Рецензент: кандидат технических наук, доцент Дмитриев В. В.

Защита состоится <<14>> июня 2017 года на заседании Государственной экзаменационной комиссии по направлению <<Механика и математическое моделирование>> в Самарском национальном исследовательском университете имени академика С. П. Королева (ауд. 516 корп. 5).

Ваши отзывы в двух экземплярах просьба высылать по адресу: 443086, г. Самара, Московское шоссе, д. 34, кафедра теоретической механики.

\vspace{6em}

Автореферат разослан <<\uline{\hspace{4em}}>>\uline{\hspace{6em}}20\uline{\hspace{2em}}г.

Секретарь Государственной экзаменационной комиссии по направлению <<Механика и математическое моделирование>>

кандидат технических наук, доцент Алексеев А. В. $\underset{\text{(подпись)}}{\text{\uline{\hspace{8em}}}}$