\documentclass[14pt,a4paper]{extarticle}
\usepackage{extsizes}
\usepackage{cmap} % для кодировки шрифтов в pdf
%\documentclass[a4paper,12pt]{article} %размер бумаги устанавливаем А4, шрифт 12пунктов
\usepackage[utf8]{inputenc}%включаем свою кодировку: koi8-r или utf8 в UNIX, cp1251 в Windows
\usepackage[T2A, T1]{fontenc}
\usepackage[english,russian]{babel}%используем русский и английский языки с переносами
\usepackage{amssymb,amsfonts,amsmath,mathtext,cite,enumerate,amsthm,mathenv} %подключаем нужные пакеты расширений
\numberwithin{equation}{section}
\usepackage{graphicx} %хотим вставлять в диплом рисунки?
\usepackage{indentfirst} % Красная строка
\graphicspath{{images/}}%путь к рисункам
\usepackage{hyperref}
\usepackage{framed}
\usepackage{color} %% это для отображения цвета в коде
\usepackage{listings} %% собственно, это и есть пакет listingsb
\usepackage{ucs}
\usepackage{latexsym}
\usepackage{bm}
\usepackage{array}
\usepackage{multirow}
\usepackage{ulem}
\usepackage{csvsimple}

\frenchspacing
%----------------ЗАГОЛОВКИ---------------------------
\usepackage{titlesec}

\titleformat{\chapter}[display]
    {\filcenter}
    {\MakeUppercase{\chaptertitlename} \thechapter}
    {8pt}
    {\bfseries}{}
 
\titleformat{\section}
    {\centering\normalsize\bfseries}
    {\thesection}
    {1em}{\MakeUppercase}
 
\titleformat{\subsection}
    {\normalsize\bfseries}
    {\thesubsection}
    {1em}{}

% Настройка вертикальных и горизонтальных отступов
%\titlespacing*{\chapter}{0pt}{-30pt}{8pt}
\titlespacing*{\section}{\parindent}{*4}{*4}
\titlespacing*{\subsection}{\parindent}{*4}{*4}
%------------------------------------------------------

%---------------------ПОЛЯ-----------------------------
\usepackage{geometry}
\geometry{left=3cm}
\geometry{right=1.5cm}
\geometry{top=2.4cm}
\geometry{bottom=2.4cm}


\makeatletter
\renewcommand{\@biblabel}[1]{#1.} % Заменяем библиографию с квадратных скобок на точку:
\makeatother

\makeatletter


\def\x@multispan#1{%
  \global\let\@tempa\@empty
  \@multicnt#1\relax
  \loop\ifnum\@multicnt>\@ne
  \xdef\@tempa{\@tempa\kern\dimen@i\hfill&\omit}%
   \advance\@multicnt\m@ne
  \repeat
  \@tempa\kern\dimen@i\hfill}


\long\def\xmulticolumn#1#2#3{%
 \omit
 \begingroup
   \def\@addamp{\if@firstamp \@firstampfalse \else
                \@preamerr 5\fi}%
  \@mkpream{#2}\@addtopreamble\@empty
  \endgroup
  \def\@sharp{#3}%
  \setbox\z@\hbox{{\@preamble}}%
\global\dimen@i\wd\z@
\global\divide\dimen@i#1\relax
 \ignorespaces
\x@multispan{#1}}
\makeatother

\linespread{1.5} % полуторный интервал
%\renewcommand{\baselinestretch}{1.5}
\renewcommand{\theenumi}{\arabic{enumi}}% Меняем везде перечисления на цифра.цифра
\renewcommand{\labelenumi}{\arabic{enumi}}% Меняем везде перечисления на цифра.цифра
\renewcommand{\theenumii}{.\arabic{enumii}}% Меняем везде перечисления на цифра.цифра
\renewcommand{\labelenumii}{\arabic{enumi}.\arabic{enumii}.}% Меняем везде перечисления на цифра.цифра
\renewcommand{\theenumiii}{.\arabic{enumiii}}% Меняем везде перечисления на цифра.цифра
\renewcommand{\labelenumiii}{\arabic{enumi}.\arabic{enumii}.\arabic{enumiii}.}% Меняем везде перечисления на цифра.цифра
%\renewcommand{\figurename}{Рисунок} 
\addto\captionsrussian{
\def\figurename{Рисунок}
\renewcommand{\refname}
    {Список использованных источников}
}
\usepackage[labelsep=space]{caption}
\DeclareCaptionLabelSeparator{bar}{ - }
\captionsetup{
  labelsep=bar
}

\usepackage{floatrow}
\DeclareFloatFont{tiny}{\tiny}
\floatsetup[table]{font=footnotesize,capposition=top}

%\usepackage{sectsty}

%\allsectionsfont{\centering}


\lstset{ %
language=Python,                 % выбор языка для подсветки (здесь это Python)
basicstyle=\scriptsize\sffamily, % размер и начертание шрифта для подсветки кода
numberstyle=\tiny,           % размер шрифта для номеров строк
stepnumber=1,                   % размер шага между двумя номерами строк
numbersep=5pt,                % как далеко отстоят номера строк от подсвечиваемого кода
backgroundcolor=\color{white}, % цвет фона подсветки - используем \usepackage{color}
showspaces=false,            % показывать или нет пробелы специальными отступами
showstringspaces=false,      % показывать или нет пробелы в строках
showtabs=false,             % показывать или нет табуляцию в строках
tabsize=2,                 % размер табуляции по умолчанию равен 2 пробелам
captionpos=t,              % позиция заголовка вверху [t] или внизу [b] 
breaklines=true,           % автоматически переносить строки (да\нет)
breakatwhitespace=false, % переносить строки только если есть пробел
escapeinside={\%*}{*)}   % если нужно добавить комментарии в коде
}

%\usepackage[explicit]{titlesec}
%\usepackage{textcase}
%\usepackage{microtype}
%\titleformat{\section}
  %{\normalfont\Large\scshape}{\large\thesection}{1em}{\textls{\MakeTextLowercase{#1}}}
\usepackage{tocloft}
\renewcommand{\cfttoctitlefont}{\hfil \normalfont\Large\bfseries\MakeUppercase}

\usepackage{inconsolata}
\newcommand{\norm}[1]{\left\lVert#1\right\rVert}%норма

\begin{document}
\begin{titlepage}

\begin{center}
\vspace{1.5em}
\small{МИНИСТЕРСТВО ОБРАЗОВАНИЯ И НАУКИ РОССИЙСКОЙ ФЕДЕРАЦИИ\\
\vspace{\baselineskip}
ФЕДЕРАЛЬНОЕ ГОСУДАРСТВЕННОЕ АВТОНОМНОЕ\\
ОБРАЗОВАТЕЛЬНОЕ УЧРЕЖДЕНИЕ ВЫСШЕГО ОБРАЗОВАНИЯ\\
<<CАМАРСКИЙ НАЦИОНАЛЬНЫЙ ИССЛЕДОВАТЕЛЬСКИЙ УНИВЕРСИТЕТ\\
ИМЕНИ АКАДЕМИКА С. П. КОРОЛЕВА>>\\}
\vspace{\baselineskip}
\end{center}

\begin{minipage}{.45\linewidth}
	\begin{flushleft}                           
	\small{УДК 002.5.004}
	\end{flushleft} 
\end{minipage}
\hfill
\begin{minipage}{.45\linewidth}
	\begin{flushright}                
	\small{На правах рукописи}
	\end{flushright} 
\end{minipage}

\vspace{3em}

\begin{center}
	Асланов Евгений Владимирович
\end{center}

\begin{center}
Удаление космического мусора путем электростатического взаимодействия с активным космическим аппаратом
\end{center}

\vspace{3em}

\begin{center}
\small{Автореферат\\
выпускной квалификационной работы магистра\\
по направлению подготовки 01.04.03 <<Механика и математическое моделирование>>\\
магистерская программа <<Математическое и компьютерное проектирование механики космических систем>>}
\end{center}

\vspace{\fill}

\begin{center}
Самара 2017
\end{center}
\end{titlepage}
\newpage

Работа выполнена в Самарском национальном исследовательском университете имени академика С. П. Королева.

Научный руководитель: доктор технических наук, профессор Асланов В. С.

Рецензент: кандидат технических наук, доцент Дмитриев В. В.

Защита состоится <<14>> июня 2017 года на заседании Государственной экзаменационной комиссии по направлению <<Механика и математическое моделирование>> в Самарском национальном исследовательском университете имени академика С. П. Королева (ауд. 516 корп. 5).

Ваши отзывы в двух экземплярах просьба высылать по адресу: 443086, г. Самара, Московское шоссе, д. 34, кафедра теоретической механики.

\vspace{6em}

Автореферат разослан <<\uline{\hspace{4em}}>>\uline{\hspace{6em}}20\uline{\hspace{2em}}г.

Секретарь Государственной экзаменационной комиссии по направлению <<Механика и математическое моделирование>>

кандидат технических наук, доцент Алексеев А. В. $\underset{\text{(подпись)}}{\text{\uline{\hspace{8em}}}}$
\end{document}