\section*{ЗАКЛЮЧЕНИЕ}
\addcontentsline{toc}{section}{Заключение}

В данной работе проведено исследования системы из активного и пассивного космического аппаратов.

В разделе \ref{SEC:3SPH} рассмотрено движение вокруг центра масс пассивного космического аппарата цилиндрической формы при постоянном расстоянии между центрами масс аппаратов. Моделирование проводилось с помощью метода многих сфер.

В разделе \ref{SEC:2SPH} проведено моделирование взаимодействия космических аппаратов, представленных материальными точками, с приложением тяги к активному аппарату. Были составлены и проанализированы тяги управления.

В разделе \ref{SEC:2SPH_MSM} рассмотрено движение вокруг центра масс пассивного космического аппарата цилиндрической формы в системе с активным космическим аппаратом. Было произведено исследование управляющей функции. Так как моделирование производилось в бессиловом поле, полученная модель пригодна для расчетов подобных взаимодействий на большом удалении от притягивающего центра или в поле действия силы тяжести (моделирование на земле).

В приложениях приведены программы для среды Wofram Mathematica.