\section*{ВВЕДЕНИЕ}
\addcontentsline{toc}{section}{Введение}

На данный момент проблема уборки космического мусора встает всё острее, так как число закончивших свои программы спутников и ступеней ракет только увеличивается.
Так же, постоянно увеличивается число объектов, считающихся космическим мусором, за счет столкновений уже ставших мусором объектов.
Всё это выводит из оборота множество орбит.

Большинство контактных способов уборки мусора либо потенциально порождают новый мусор (гарпуны, взрывы), либо слишком сложны в управлении (сети).
Один из бесконтактных способов, находящийся в разработке, будет рассмотрен в этой работе.

В работе рассматривается движение вокруг центра масс пассивного космического аппарата цилиндрической формы при постоянном расстоянии между центрами масс аппаратов. Моделирование проводится с помощью метода многих сфер.
Так же будет рассмотрено моделирование взаимодействия космических аппаратов, представленных материальными точками, с приложением тяги к активному аппарату. 
Будут составлены и проанализированы тяги управления.
Будут рассмотрены движение вокруг центра масс пассивного космического аппарата цилиндрической формы в системе с активным космическим аппаратом. Будет произведено исследование управляющей функции.