\section*{Общая характеристика работы}
\textbf{Актуальность темы выпускной квалификационной работы}

На данный момент проблема уборки космического мусора встает всё острее, так как число закончивших свои программы спутников и ступеней ракет только увеличивается.
Так же, постоянно увеличивается число объектов, считающихся космическим мусором, за счет столкновений уже ставших мусором объектов.
Всё это выводит из оборота множество орбит.

Большинство контактных способов уборки мусора либо потенциально порождают новый мусор (гарпуны, взрывы), либо слишком сложны в управлении (сети).
Один из бесконтактных способов, находящийся в разработке, рассмотрен в этой работе.

\textbf{Объект и предмет исследования}

Объектом исследования является система из пассивного и активного космических аппаратов при электростатическом взаимодействии между ними.

\textbf{Цели и задачи исследования}

Цель работы – исследовать применение метода многих сфер  для бесконтактного увода с орбиты космического мусора.
Также показать целесообразность применения метода многих сфер вместо постоянных зарядов при моделировании движения относительно центра масс и рассмотреть управление для такой модели.

Основная задача с использованием метода многих сфер и уравнения Лагранжа второго рода было провести моделирование для трёх случаев взаимодействия аппаратов:
\begin{itemize}
	\item При моделировании движения космического аппарата цилиндрической формы вокруг центра масс с активным спутником при поддержании постоянного расстояния между центрами масс двух космических аппаратов методом многих сфер,
	\item При моделировании движения двух космических аппаратов как двух материальных точек при действии тяги на одном из них,
	\item При моделировании движения пассивного космического аппарата цилиндрической формы и активного космического аппарата методом многих сфер.
\end{itemize}