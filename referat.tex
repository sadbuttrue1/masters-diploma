\section*{Реферат}
\textbf{Выпускная квалификационная работа магистра}: 48 страниц, 40 рисунков, 4 источника, 3 приложения. 

\textbf{Презентация}: 20 слайдов PDF.

\hspace{1pt}

ДИФФЕРЕНЦИАЛЬНОЕ УРАВНЕНИЕ ДВИЖЕНИЯ, УРАВНЕНИЕ ЛАГРАНЖА ВТОРОГО РОДА, КУЛОНОВСКОЕ ВЗАИМОДЕЙСТВИЕ, МЕТОД МНОГИХ СФЕР, УПРАВЛЕНИЕ ДВИЖЕНИЕМ

\hspace{1pt}

В работе рассматривается движение системы, состоящей из пассивного космического аппарата (верхней ступени ракеты) и активного космического аппарата, при кулоновском взаимодействии между объектами системы.
Такая связка может быть применена для бесконтактной уборки космического мусора.

Цель работы – исследовать применение метода многих сфер  для бесконтактного увода с орбиты космического мусора.
Также показать целесообразность применения метода многих сфер вместо постоянных зарядов при моделировании движения относительно центра масс и рассмотреть управление для такой модели.

С использование метода многих сфер и уравнения Лагранжа второго рода было проведено моделирование для трёх случаев взаимодействия аппаратов:
\begin{itemize}
	\item При моделировании движения космического аппарата цилиндрической формы вокруг центра масс с активным спутником при поддержании постоянного расстояния между центрами масс двух космических аппаратов методом многих сфер,
	\item При моделировании движения двух космических аппаратов как двух материальных точек при действии тяги на одном из них,
	\item При моделировании движения пассивного космического аппарата цилиндрической формы и активного космического аппарата методом многих сфер.
\end{itemize}
При помощи системы Wolfram Mathematica произведено численное моделирование, решены системы уравнений для рассматриваемых систем и графически приведены результаты моделирования.